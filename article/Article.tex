%
% Complete documentation on the extended LaTeX markup used for Insight
% documentation is available in ``Documenting Insight'', which is part
% of the standard documentation for Insight.  It may be found online
% at:
%
%     http://www.itk.org/

\documentclass{InsightArticle}


%%%%%%%%%%%%%%%%%%%%%%%%%%%%%%%%%%%%%%%%%%%%%%%%%%%%%%%%%%%%%%%%%%
%
%  hyperref should be the last package to be loaded.
%
%%%%%%%%%%%%%%%%%%%%%%%%%%%%%%%%%%%%%%%%%%%%%%%%%%%%%%%%%%%%%%%%%%
\usepackage[dvips,
bookmarks,
bookmarksopen,
backref,
colorlinks,linkcolor={blue},citecolor={blue},urlcolor={blue},
]{hyperref}
% to be able to use options in graphics
\usepackage{graphicx}
% for pseudo code
\usepackage{listings}
% subfigures
\usepackage{subfigure}


%  This is a template for Papers to the Insight Journal. 
%  It is comparable to a technical report format.

% The title should be descriptive enough for people to be able to find
% the relevant document. 
\title{CITK - an architecture and examples of CUDA enabled ITK filters}

% Increment the release number whenever significant changes are made.
% The author and/or editor can define 'significant' however they like.
\release{0.00}

% At minimum, give your name and an email address.  You can include a
% snail-mail address if you like.
\author{Richard Beare{$^1$}, Daniel Micevski, Chris Share\\Luke Parkinson, Phillip Ward, Mike Kuiper{$^2$}}
\authoraddress{{$^1$}Richard.Beare@monash.edu, Monash University, Melbourne, Australia\\
{$^2$}mike@vpac.org, Victorian Partnership for Advanced Computing, Melbourne, Australia.}

\begin{document}
\maketitle

\ifhtml
\chapter*{Front Matter\label{front}}
\fi


\begin{abstract}
\noindent
There is great interest in the use of graphics processing units (GPU)
for general purpose applications because the highly parallel
architectures used in GPUs offer the potential for huge performance
increases. The use of GPUs in image analysis applications has been
under investigation for a number of years. This article describes
modifications to the InsightToolkit (ITK) that provide a simple
architecture for transparent use of GPU enabled filters and examples
of how to write GPU enabled filters using the NVIDIA CUDA tools.
\end{abstract}

\tableofcontents

\section{Introduction}
Data must be resident in GPU device memory in order to be processed by
the GPU. In order for an ITK filter to be accelerated using GPUs an
image must be copied to the device memory and the result copied back
if the next filter is not GPU enabled. Copying between host and device
memory is quite slow and can easily offset any benefits achieved by
faster GPU processing. It is therefore essential that redundant copies
between host and device memory are eliminated. It is also desirable
that new, GPU enabled, filters can be included in applications without
changing programming style.

This article describes a simple modification to the itk::Image class
that allows transparent use of CUDA enabled filters. A range of
standard filters have been implemented and extensive testing
performed.

\section{CITK Architecture}
The aim of the architecture outline below was to allow GPU enabled
filters to be included in an application without change of programming
style or losing performance via redundant host to device memory copies.

A number of architectures were considered. These were derived from online discussions and small samples of code available online:
\begin{itemize}
\item Break the pipeline at the beginning of filter execution by
  copying data to device memory, processing, and then copying back
  after execution completes. This isolates the GPU code from the rest
  of the pipeline and requires no change to ITK infrastructure, but
  introduces redundant copies if subsequent filters are GPU enabled.
\item Include interface objects between filters in the pipeline to
  manage copying. This can eliminate redundant copies but requires
  that the programmer be aware of which filters are GPU enabled. There
  is also a minor change of programming style.
\end{itemize}

Neither of these options require a modification to core ITK classes.

The approach used in CITK does require a modification to core ITK
classes, but has a number of advantages. The same approach has since
been outlined on the ITK Wiki.

The fundamental component of the pipeline is the ITK Image
class. Within this class is a pixel container called {\em
  ImportImageContainer}, used to manage the image data. CITK includes
a substitute pixel container named {\em
  CudaImportImageContainer}. This pixel container has all the same
functionality of the {\em ImportImageContainer} which results in full
compatibility with existing ITK components.

The {\em CudaImportImageContainer} not only managed the image data on
the host, but it also managed the image data on the device. When a
standard filter requested the image data, such as through an iterator,
the {\em CudaImportImageContainer} checks whether the most up to date
image is on the device or the host. If it is on the device, it is
copied back onto the host. This data is then supplied to the
user. Similarly when a GPU filter requests the image data, the {\em
  CudaImportImageContainer} would check where the most up to date
image is, and copy it to the device if required.

The {\em CudaImportImageContainer} can track where the most up to the
date image is by which set command was used last, and assumes the data
is modified when a standard iterator requests it.

The result of this is memory transfers are only performed when
required and are completed transparent to both the developer and the
user. This leaves all the responsibility on the architect, rather than
the developer or the user such as in the other attempts.

\section{Anatomy of a CUDA enabled filter}
\subsection{Image parameters}
Different image operations require different image information. Number
of pixels and image dimensions are all available in the normal way via
GetSize, GetNumberOfPixels and related methods.

\subsection{Memory management}

Your CUDA kernel function should return a pointer to the output image as a global device pointer:

{\tt typename TOutputImage::PixelType * ptr;}

A pointer to input image memory is retrieved using:

{\tt input->GetDevicePointer();}

After the filter has completed, the output pointer must be given to
the output image to be passed down the pipeline:

{\tt output->GetPixelContainer()->SetDevicePointer(ptr, N, true);}

In this example the {\tt ptr} is the pointer to output memory and the 'N'
is the number of pixels in this image. The 'true' value is very
important. This tells the output image to free the device memory once
it is finished with it. WARNING: Setting this to false can lead to
memory leaks on the device.



\subsection{Templated kernel file}

%\subsection{}

\bibliographystyle{plain}
%\bibliography{local,InsightJournal}
\bibliography{InsightJournal}
\nocite{ITKSoftwareGuide}

\end{document}

