%
% Complete documentation on the extended LaTeX markup used for Insight
% documentation is available in ``Documenting Insight'', which is part
% of the standard documentation for Insight.  It may be found online
% at:
%
%     http://www.itk.org/

\documentclass{InsightArticle}


%%%%%%%%%%%%%%%%%%%%%%%%%%%%%%%%%%%%%%%%%%%%%%%%%%%%%%%%%%%%%%%%%%
%
%  hyperref should be the last package to be loaded.
%
%%%%%%%%%%%%%%%%%%%%%%%%%%%%%%%%%%%%%%%%%%%%%%%%%%%%%%%%%%%%%%%%%%
\usepackage[dvips,
bookmarks,
bookmarksopen,
backref,
colorlinks,linkcolor={blue},citecolor={blue},urlcolor={blue},
]{hyperref}
% to be able to use options in graphics
\usepackage{graphicx}
% for pseudo code
\usepackage{listings}
% subfigures
\usepackage{subfigure}


%  This is a template for Papers to the Insight Journal. 
%  It is comparable to a technical report format.

% The title should be descriptive enough for people to be able to find
% the relevant document. 
\title{CITK - an architecture and examples of CUDA enabled ITK filters}

% Increment the release number whenever significant changes are made.
% The author and/or editor can define 'significant' however they like.
\release{0.00}

% At minimum, give your name and an email address.  You can include a
% snail-mail address if you like.
\author{Richard Beare{$^1$}, Daniel Micevski, Chris Share\\Luke Parkinson, Phillip Ward, Mike Kuiper{$^2$}}
\authoraddress{{$^1$}Richard.Beare@monash.edu, Monash University, Melbourne, Australia\\
{$^2$}mike@vpac.org, Victorian Partnership for Advanced Computing, Melbourne, Australia.}

\begin{document}
\maketitle

\ifhtml
\chapter*{Front Matter\label{front}}
\fi


\begin{abstract}
\noindent
There is great interest in the use of graphics processing units (GPU)
for general purpose applications because the highly parallel
architectures used in GPUs offer the potential for huge performance
increases. The use of GPUs in image analysis applications has been
under investigation for a number of years. This article describes
modifications to the InsightToolkit (ITK) that provide a simple
architecture for transparent use of GPU enabled filters and examples
of how to write GPU enabled filters using the NVIDIA CUDA tools.

This work was performed between late 2009 and early 2010 and is being
published as modifications to ITK 3.20. It is hoped that publication will
help inform more general development of GPU support in ITK 4.0 and
facilitate experimentation by users requiring functionality of 3.20 or
wishing to pursue CUDA based developments.
\end{abstract}

\tableofcontents

\section{Introduction}
Data must be resident in GPU device memory in order to be processed by
the GPU. In order for an ITK filter to be accelerated using GPUs an
image must be copied to the device memory and the result copied back
if the next filter is not GPU enabled. Copying between host and device
memory is quite slow and can easily offset any benefits achieved by
faster GPU processing. It is therefore essential that redundant copies
between host and device memory are eliminated. It is also desirable
that new, GPU enabled, filters can be included in applications without
changing programming style.

This article describes a simple modification to the itk::Image class
that allows transparent use of CUDA enabled filters. A range of
standard filters have been implemented and extensive testing
performed.

\section{CITK Architecture}
The aim of the architecture outline below was to allow GPU enabled
filters to be included in an application without change of programming
style or losing performance via redundant host to device memory copies. 

A number of architectures were considered. These were derived from
online discussions and small samples of code available online:
\begin{itemize}
\item Break the pipeline at the beginning of filter execution by
  copying data to device memory, processing, and then copying back
  after execution completes. This isolates the GPU code from the rest
  of the pipeline and requires no change to ITK infrastructure, but
  introduces redundant copies if subsequent filters are GPU enabled.
\item Include interface objects between filters in the pipeline to
  manage copying. This can eliminate redundant copies but requires
  that the programmer be aware of which filters are GPU enabled. There
  is also a minor change of programming style.
\end{itemize}

Neither of these options require a modification to core ITK classes.

The approach used in CITK does require a modification to core ITK
classes, but has a number of advantages. A similar approach has since
been outlined on the ITK Wiki {\url http://www.cmake.org/Wiki/ITK_Release_4/GPU_Acceleration}.

The fundamental component of the pipeline is the {\em itk::Image}
class. Within this class is a pixel container called {\em
  ImportImageContainer}, used to manage the image data. CITK includes
a substitute pixel container named {\em
  CudaImportImageContainer}. This pixel container has all the same
functionality of the {\em ImportImageContainer} which results in full
compatibility with existing ITK components.

The {\em CudaImportImageContainer} manages image data on both the host
and device. When a standard filter requests the image data, such as
through an iterator, the {\em CudaImportImageContainer} checks whether
the most up to date image is on the device or the host. If it is on
the device, it is copied back onto the host. This data is then
supplied to the user. Similarly when a GPU filter requests the image
data, the {\em CudaImportImageContainer} would check where the most up
to date image is, and copy it to the device if required.

The {\em CudaImportImageContainer} can track where the most up to the
date image is by which set command was used last, and assumes the data
is modified when a standard iterator requests it.

The result of this is memory transfers are only performed when
required and are completed transparent to both the developer and the
user. This leaves all the responsibility on the architect, rather than
the developer or the user such as in the other attempts.

\subsection{Weaknesses}
\begin{itemize}
\item This framework only supports Cuda, and not OpenCL. The ITK 4.0
proposal supports OpenCL. Limitations of the Cuda development
environment mean that even Cuda integration is less complete than
hoped, with significant changes to compilation processes being
necessary (see below).
\item A copy between host and device always results in the source of
the copy being considered redundant. This could be inefficient in some
cases. The problem is correctly dealing with identical copies on both
host and device. If, for example, a pipeline is branched such that one
branch in on GPU and the other on CPU, then the branch point is likely
to become a source of redundant copies.
\item The need to copy between device and host memory breaks some of
the usual assumptions, leading to some ugly use of {\em mutable}
declarations in CudaImportImageContainer.
\item Morphology filters included in the package use texture memory
but are not as generic as the standard ITK versions.
\end{itemize}

\section{Installation and building}
\subsection{Cuda compiler and software development kit}
This framework requires Cuda 3.2 and the SDK.
\subsection{Patch ITK 3.20}
The code distributed with this article includes a patch to modify ITK 3.20. This can be applied as follows:

\begin{itemize}
\item {\url http://voxel.dl.sourceforge.net/sourceforge/itk/InsightToolkit-3.20.0.tar.gz}
\item extract
\item cd ITK-3.20
\item patch -p0 < path/to/patch_3.20.dif
\end{itemize}

Alternatviely, this code may currently be retrieved via git and gerrit, as follows:
\begin{itemize}
\item git clone git://itk.org/ITK.git
\item cd ITK
\item git checkout v3.20.0
\item git pull http://review.source.kitware.com/p/ITK refs/changes/52/1452/1
\end{itemize}

\subsection{Build and install modified ITK}
There are many options available when building ITK. This process has
been tested under Linux and there are a number of changes to defaults
required to avoid limitations to the cuda development tools. 
\begin{itemize}
  \item Specify location of Cuda SDK.
  \item Turn off SSE options for VNL - advanced/VNL. This avoids errors caused by multiple inclusion of SSE files.
\end{itemize}

\subsection{Build examples}
\subsection{Workarounds for nvcc weaknesses/bugs}
\begin{itemize}
  \item multiple include of SSE files - turn off SSE options for VNL - advanced/VNL.
\end{itemize}


\subsection{Changes to standard processes for building ITK applications}
Typical application development in ITK utilizes templates and generic
programming and therefore does not require that the developer track
new object code dependencies when adding new filters. In principle the
same procedure should be possible when using CUDA enabled devices by
compiling all application code with nvcc, leading to non-cuda code
being compiled with the host c++ compiler and cuda code being compiled
with cuda compilers. This would also allow useful templating of cuda
kernels, leading to a relatively seamless integration with traditional
ITK development. Unfortunately the current generation of CUDA tools is
not able to cope with c++ of the complexity used in ITK. It is
therefore necessary to compile cuda kernels separately, which means
the developer must specify the correct object dependencies to the
linker. Examples of this can be seen in the CMake files included with
this article.

Alternatives, such as compiling all cuda kernels into a library, are
feasible but haven't been tested during this development.

\section{Anatomy of a CUDA enabled filter}
\subsection{Image parameters}
Different image operations require different image information. Number
of pixels and image dimensions are all available in the normal way via
GetSize, GetNumberOfPixels and related methods.

\subsection{Memory management}

Two base classes have been provided to handle standard filter memory
management - {\em CudaInPlaceImageFilter} and {\em
CudaImageToImageFilter}. These filters allow the standard allocation
structure to be used via {\em this->AllocateOutputs()}. Explicit
allocation of device memory can be achieved using
Image->AllocateGPU().

\subsection{Templated kernel files}
ITK filters are generic with respect to pixel type and dimension.



\subsection{Templated kernel file}

%\subsection{}

\bibliographystyle{plain}
%\bibliography{local,InsightJournal}
\bibliography{InsightJournal}
\nocite{ITKSoftwareGuide}

\end{document}

