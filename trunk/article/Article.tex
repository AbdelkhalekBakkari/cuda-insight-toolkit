%
% Complete documentation on the extended LaTeX markup used for Insight
% documentation is available in ``Documenting Insight'', which is part
% of the standard documentation for Insight.  It may be found online
% at:
%
%     http://www.itk.org/

\documentclass{InsightArticle}


%%%%%%%%%%%%%%%%%%%%%%%%%%%%%%%%%%%%%%%%%%%%%%%%%%%%%%%%%%%%%%%%%%
%
%  hyperref should be the last package to be loaded.
%
%%%%%%%%%%%%%%%%%%%%%%%%%%%%%%%%%%%%%%%%%%%%%%%%%%%%%%%%%%%%%%%%%%
\usepackage[dvips,
bookmarks,
bookmarksopen,
backref,
colorlinks,linkcolor={blue},citecolor={blue},urlcolor={blue},
]{hyperref}
% to be able to use options in graphics
\usepackage{graphicx}
% for pseudo code
\usepackage{listings}
% subfigures
%\usepackage{subfigure}


%  This is a template for Papers to the Insight Journal. 
%  It is comparable to a technical report format.

% The title should be descriptive enough for people to be able to find
% the relevant document. 
\title{CITK - an architecture and examples of CUDA enabled ITK filters}

% Increment the release number whenever significant changes are made.
% The author and/or editor can define 'significant' however they like.
\release{0.00}

% At minimum, give your name and an email address.  You can include a
% snail-mail address if you like.
\author{Richard Beare{$^1$}, Daniel Micevski, Chris Share\\Luke
  Parkinson, Phil Ward, Wojtek Goscinski{$1$}, Mike Kuiper{$^2$}}
\authoraddress{{$^1$}Richard.Beare@monash.edu, Monash University, Melbourne, Australia\\
{$^2$}mike@vpac.org, Victorian Partnership for Advanced Computing, Melbourne, Australia.}

\begin{document}
\maketitle

\ifhtml
\chapter*{Front Matter\label{front}}
\fi


\begin{abstract}
\noindent
There is great interest in the use of graphics processing units (GPU)
for general purpose applications because the highly parallel
architectures used in GPUs offer the potential for huge performance
increases. The use of GPUs in image analysis applications has been
under investigation for a number of years. This article describes
modifications to the InsightToolkit (ITK) that provide a simple
architecture for transparent use of GPU enabled filters and examples
of how to write GPU enabled filters using the NVIDIA CUDA tools.

This work was performed between late 2009 and early 2010 and is being
published as modifications to ITK 3.20. It is hoped that publication will
help inform development of more general GPU support in ITK 4.0 and
facilitate experimentation by users requiring functionality of 3.20 or
wishing to pursue CUDA based developments.
\end{abstract}

\tableofcontents

\section{Introduction}
Data must be resident in GPU device memory in order to be processed by
the GPU. In order for an ITK filter to be accelerated using GPUs an
image must be copied to the device memory and the result copied back
if the next filter is not GPU enabled. Copying between host and device
memory is quite slow and can easily offset any benefits achieved by
faster GPU processing. It is therefore essential that redundant copies
between host and device memory are eliminated. It is also desirable
that new, GPU enabled, filters can be included in applications without
changing programming style.

This article describes a simple modification to the itk::Image class
that allows transparent use of CUDA enabled filters. A range of
standard filters have been implemented and extensive testing
performed.

\section{CITK Architecture}
The aim of the architecture outlined below was to allow GPU enabled
filters to be included in an application without change of programming
style or losing performance via redundant host to device memory copies. 

A number of architectures were considered. These were derived from
online discussions and small samples of code available online:
\begin{itemize}
\item Break the pipeline at the beginning of filter execution by
  copying data to device memory, processing, and then copying back
  after execution completes. This isolates the GPU code from the rest
  of the pipeline and requires no change to ITK infrastructure, but
  introduces redundant copies if subsequent filters are GPU enabled.
\item Include interface objects between filters in the pipeline to
  manage copying. This can eliminate redundant copies but requires
  that the programmer be aware of which filters are GPU enabled. There
  is also a minor change of programming style.
\end{itemize}

Neither of these options require a modification to core ITK classes.

The approach used in CITK does require a modification to core ITK
classes, but has a number of advantages. A similar approach has since
been outlined on the ITK Wiki \url{http://www.cmake.org/Wiki/ITK\_Release\_4/GPU\_Acceleration}.

The fundamental component of the pipeline is the {\em itk::Image}
class. Within this class is a pixel container called {\em
  ImportImageContainer}, used to manage the image data. CITK includes
a substitute pixel container named {\em
  CudaImportImageContainer}. This pixel container has all the same
functionality of the {\em ImportImageContainer} which results in full
compatibility with existing ITK components.

The {\em CudaImportImageContainer} manages image data on both the host
and device. When a standard filter requests the image data, such as
through an iterator, the {\em CudaImportImageContainer} checks whether
the most up to date image is on the device or the host. If it is on
the device, it is copied back onto the host. This data is then
supplied to the user. Similarly when a GPU filter requests the image
data, the {\em CudaImportImageContainer} would check where the most up
to date image is, and copy it to the device if required.

The {\em CudaImportImageContainer} can track where the most up to the
date image is by which set command was used last, and assumes the data
is modified when a standard iterator requests it.

The result of this is memory transfers are only performed when
required and are completed transparent to both the developer and the
user. This leaves all the responsibility on the architect, rather than
the developer or the user such as in the other attempts.

Some exceptions have been uncovered during this development. Minor
changes have also been made to the {\em AllocateOutputs} method of the
{\em itkInPlaceImageFilter} to support pipelines that connect in
place, multi-threaded CPU filters to a CUDA filter. Other exceptions
are discussed below.
 
\subsection{Weaknesses}
\begin{itemize}
\item This framework only supports CUDA, and not OpenCL. The ITK 4.0
proposal supports OpenCL. Limitations of the CUDA development
environment mean that even CUDA integration is less complete than
hoped, with significant changes to compilation processes being
necessary (see below).
\item A copy between host and device always results in the source of
the copy being considered redundant. This could be inefficient in some
cases. The problem is correctly dealing with identical copies on both
host and device. If, for example, a pipeline is branched such that one
branch in on GPU and the other on CPU, then the branch point is likely
to become a source of redundant copies.
\item The need to copy between device and host memory breaks some of
the usual assumptions, leading to some ugly use of {\em mutable}
declarations in CudaImportImageContainer.
\item Morphology filters included in the package use texture memory
but are not as generic as the standard ITK versions.
\item ITK filters are able to provide their own implementations of key
methods, such as {\em AllocateOutputs}. This can lead to problems when
connecting CUDA enabled components to a CPU pipeline. This problem has
been observed in the {\em itkStatisticsImageFilter}, which is a
multi-threaded {\em ImageToImageFilter} with its own AllocateOutputs
method passing input through to output. This bypasses the trigger to
copy device memory back to the host, leading to incorrect
results. Other filters with unusual structure are likely to cause
problems. The simple fix for such filters is to call {\em
  GetBufferPointer} in the {\em AllocateOutputs} method.
\end{itemize}

\section{Installation and building}
\subsection{CUDA compiler and software development kit}
This framework requires CUDA 3.2 and the SDK. The {\em thrust} library
is also included in the examples to implement more complex components
of some sample filters.
\subsection{Fetch this contribution from google code}
The source code is included with this article and is also available from google code:
\url{http://code.google.com/p/cuda-insight-toolkit/}. The patch for ITK is included.

\subsection{Patch ITK 3.20}
The code distributed with this article includes a patch to modify ITK
3.20, called {\em patch.3.20.0.dif}. This can be applied as follows:

\begin{itemize}
\item fetch \url{http://voxel.dl.sourceforge.net/sourceforge/itk/InsightToolkit-3.20.0.tar.gz}
\item extract
\item cd ITK-3.20
\item patch -p0 $<$ path/to/cuda-insight-toolkit/patch.3.20.0.dif
\end{itemize}

Alternatviely, this code may currently be retrieved via git, as follows:
\begin{itemize}
\item git clone git://github.com/richardbeare/ITK.git
\item cd ITK
\item git checkout v3.20.0\_cuda
\end{itemize}

\subsection{Build and install modified ITK}
There are many options available when building ITK. This process has
been tested under Linux and there are a number of changes to defaults
required to avoid limitations to the CUDA development tools. 
\begin{itemize}
  \item Specify location of CUDA SDK. Note that if there is trouble locating libcutil.a, it may be set explicitly under advanced options - CUDA\_CUT\_LIBRARY.
  \item Turn off SSE options for VNL - see advanced/VNL. This avoids errors caused by multiple inclusion of SSE files.
  \item Enable ITK\_USE\_REVIEW under advanced options, in order to build all of the tests.
\end{itemize}

\subsection{Build examples}
The Examples subdirectory in the citk distribution includes a
CMakeFile for building all examples and running tests. Location of the
patched ITK, nvcc and CUDA SDK must be provided during configuration.

\subsection{Changes to standard processes for building ITK applications}
Typical application development in ITK utilizes templates and generic
programming and therefore does not require that the developer track
new object code dependencies when adding new filters. In principle the
same procedure should be possible when using CUDA enabled devices by
compiling all application code with nvcc, leading to non-CUDA code
being compiled with the host c++ compiler and CUDA code being compiled
with CUDA compilers. This would also allow useful templating of CUDA
kernels, leading to a relatively seamless integration with traditional
ITK development. Unfortunately the current generation of CUDA tools is
not able to cope with c++ of the complexity used in ITK. It is
therefore necessary to compile CUDA kernels separately, which means
the developer must specify the correct object dependencies to the
linker. Examples of this can be seen in the CMake files included with
this article.

This approach also implies that the CUDA kernels need to be compiled
for the appropriate types, and this is currently achieved using a set
of macros supporting a limited range of input and output
types. Compiling application code with nvcc would eliminate these
macros.

Alternatives, such as compiling all CUDA kernels into a library, are
feasible but haven't been tested during this development.



\section{Anatomy of a CUDA enabled filter}
CUDA enabled filters can look very like a conventional ITK filter,
with the main difference being a call to a CUDA kernel function from
within the {\em GenerateData} method. CUDA-enabled filters should
never have {\em ThreadedGenerateData} methods as threading is provided
within the CUDA portion.

Pointers to device memory are obtained using the {\em
  GetDevicePointer} method and are passed to CUDA kernel functions.

\subsection{Memory management}
Two base classes have been provided to handle standard filter memory
management - {\em CudaInPlaceImageFilter} and {\em
CudaImageToImageFilter}. These filters allow the standard allocation
structure to be used via {\em this->AllocateOutputs()}. Explicit
allocation of device memory can be achieved using
{\em Image->AllocateGPU()}

\subsection{Templated kernel files}
ITK filters are generic with respect to pixel type and dimension and
hence CUDA kernels should offer the same flexibility. This is not
currently possible. The structure outlined in this section is the best
approximation we have been able to achieve and we hope it will evolve
to be better integrated with ITK as the CUDA tools improve.

Each kernel function has a declaration - e.g. CudaAddImageFilterKernel.h contains
\begin{verbatim}
template <class T, class S> extern
void AddImageKernelFunction(const T* input1, const T* input2, S* output, unsigned int N);
\end{verbatim}

The corresponding CudaAddImageFilterKernel.cu file contains:
{\small
\begin{verbatim}
template <class T, class S>
__global__ void AddImageKernel(T *output, const S *input, int N)
{
  int idx = blockIdx.x * blockDim.x + threadIdx.x;
  if (idx<N) 
    {
    output[idx] += input[idx];
    }
}

template <class T, class S>
__global__ void AddImageKernel(T *output, const S *input1, const S* input2, int N)
{
   int idx = blockIdx.x * blockDim.x + threadIdx.x;
   if (idx<N) 
     {
     output[idx] = input1[idx] + input2[idx];
     }
}

template <class T, class S>
void AddImageKernelFunction(const T* input1, const T* input2, S* output, unsigned int N)
{


    // Compute execution configuration 
    int blockSize = 128;
    int nBlocks = N/blockSize + (N%blockSize == 0?0:1);

    // Call kernels optimized for in place filtering
    if (output == input1)
      AddImageKernel <<< nBlocks, blockSize >>> (output, input2, N);
    else
      AddImageKernel <<< nBlocks, blockSize >>> (output, input1, input2, N);
}

#define THISTYPE float
template void AddImageKernelFunction<THISTYPE, THISTYPE>(const THISTYPE * input1, 
                                                         const THISTYPE * input2, 
                                                         THISTYPE * output, unsigned int N);
#undef THISTYPE
#define THISTYPE int
template void AddImageKernelFunction<THISTYPE, THISTYPE>(const THISTYPE * input1, 
                                                         const THISTYPE * input2,
                                                         THISTYPE *output, unsigned int N);
#undef THISTYPE

#define THISTYPE short
template void AddImageKernelFunction<THISTYPE, THISTYPE>(const THISTYPE * input1,
                                                         const THISTYPE * input2,
                                                         THISTYPE *output, unsigned int N);
#undef THISTYPE

#define THISTYPE unsigned char
template void AddImageKernelFunction<THISTYPE, THISTYPE>(const THISTYPE * input1, 
                                                         const THISTYPE * input2,
                                                         THISTYPE *output, unsigned int N);
#undef THISTYPE

\end{verbatim}
}
This is a simple structure that allows a number kernels to be precompiled for different voxel types.

\subsection{Using {\em thrust} algorithms}
The thrust project, \url{http://code.google.com/p/thrust/}, is a
source of templated, CUDA-enabled algorithms. The
CudaStatisticsImageFilter makes use of these algorithms. It is also
possible to implement simple arithmetic filters using the thrust {\em
  transform} algorithm (leading to more elegant code), but preliminary
tests suggest a significant loss in performance. There are
examples of thrust-based arithmetic in several sample filters that can
be switched on with a cmake option.

\section{Testing}
The online testing within the Insight Journal does not support CUDA
and therefore cannot be used to test this contribution.
\subsection{ITK}
Changes to ITK classes have been tested using standard ITK tests,
producing the same results as an unmodified ITK.
\subsection{Testing CUDA filters}
A range of simple CUDA enabled filters have been developed and
compared to the CPU equivalents. CMake based test are included in this
contribution.

\section{Performance}
Improved computational performance is the reason for interest in GPU
imaging applications. However there are many stories of how difficult
this is to achieve in practice, and similar difficulties are likely to
be experienced in the imaging domain. Some of the difficulties we
forsee are:
\begin{itemize}
\item A real imaging application is likely to utilize a large number
  of ITK filters. It is likely to be a long time before a significant
  portion of ITK is CUDA (or GPU) enabled. Therefore a developer will
  only experience a nett performance gain if the CUDA enabled filter
  offers sufficient speedup to offset memory transfer costs and a
  small proportion of the application is particularly time consuming.
\item Filters that are easy to port to the GPU tend to be fast on CPU
  anyway, and typically don't represent a large proportion of
  application time. Examples include all voxel-wise operations, such
  as masking and arithmetic.
\item Many potentially time consuming operations, such as filtering
  with large kernels, have been highly optimised for CPU
  implementation. It is important the comparisons are made with these
  optimised CPU implementations.
\end{itemize}

We won't discuss the mechanics of CUDA performance profiling in this
article - there are many resources available online.

One point worth noting when testing CUDA enabled ITK filters is that
there is a per-process cost associated with running a CUDA-enabled
application. This cost appears to relate to a number of things,
including loading libraries and initializing the device. This cost can
be very significant - as much as 2.5 seconds on one of our test
machines - and can give an exagerated negative impression of the  filter
performance. 

Finally, some positive performance results. These tests were carried
out using a Tesla T10 in a 16 core, 2261.051MHz, Intel L5520  Xeon:
\begin{itemize}
\item Performance improvement of 190 times observed with simple arithmetic, such as adding or subtracting constants from
  images. See {\em simple\_perf\_test.cxx} for details.
\item Image filtering with kernels, for example simple means, offer 32
  times speedup for 3d kernels, radius 10 voxels on images size $500
  \times 500 \times 500$ and 50 times speedup for kernels radius
  20.. This is potentially interesting for applications requiring
  kernels of specific shape, because most accelerated CPU schemes can
  only implement a restricted range of shape kernels. Neither CPU nor
  GPU examples exploits redundancy in this test.
\end{itemize}


\section{Conclusions}
We have provided some simple modifications to ITK infrastructure that
allow integration of CUDA enabled filters with ITK applications, and
provided a number of examples and validation tests. We hope this
framework will encourage immediate experimentation with GPU filters
and inform some of the GPU development scheduled for ITK 4.

\bibliographystyle{plain}
%\bibliography{local,InsightJournal}
\bibliography{InsightJournal}
\nocite{ITKSoftwareGuide}

\end{document}

